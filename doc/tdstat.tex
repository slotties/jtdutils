\subsection{tdstat}

Another analystic tool is tdstat.
It helps giving statistics about thread dumps.
These (shortened) thread dumps for example:
\begin{lstlisting}
<dump id="0">
<thread name="Thread1" state="OBJ_WAIT" ...>...</thread>
<thread name="Thread2" state="OBJ_WAIT" ...>...</thread>
<thread name="Thread3" state="RUN" ...>...</thread>
</dump>
<dump id="2">
<thread name="Thread1" state="OBJ_WAIT" ...>...</thread>
<thread name="Thread2" state="OBJ_WAIT" ...>...</thread>
<thread name="Thread3" state="OBJ_WAIT" ...>...</thread>
</dump>
\end{lstlisting}

Allow tdstat to render statistics for these dumps:
\begin{lstlisting}
tdstrip mylog | tdstat
\end{lstlisting}

You'll receive:
\begin{lstlisting}
                                   |  RUN | OBJ_WAI | WAI_MON | WAI_CON | SLEEP
-----------------------------------|------|---------|---------|---------|------
D:                               0      1         2         0         0      0
D:                               1      0         3         0         0      0
\end{lstlisting}

You see both dumps and the amount of threads for each state.
This itself might not be very helpful, so there's the possibility to define filters.
A filter is a rule that receives its own line for each dump in the statistic and displays all threads that match this rule.
A filter is defined like this:
\begin{lstlisting}
tdstat -f value,method
\end{lstlisting}

e,g.:
\begin{lstlisting}
tdstat -f Thread3,e
\end{lstlisting}

Value is the value to filter for and method is the filtering method used:
\begin{description}
\item[e] exact match
\item[s] starting with
\item[c] contains
\item[r] regular expression
\end{description}

As you see these are the same methods/types other tools like tdgrep use.
Currently the filtering is just done for thread names, so the value will be matched against the thread name.
Let's apply the filter I used for example some lines above:
\begin{lstlisting}
                                   |  RUN | OBJ_WAI | WAI_MON | WAI_CON | SLEEP
-----------------------------------|------|---------|---------|---------|------
D:                               0      1         2         0         0      0
F:                          hread3      1         0         0         0      0
D:                               1      0         3         0         0      0
F:                          hread3      0         1         0         0      0
\end{lstlisting}

See?
Using filters it's easy to check if special thread types, e.g. request threads, queue up without using tdlocks.
When you've determined a massive amount of threads you can use tdlocks to go deeper into analysis.
