\subsection{tdfilter}

One of the filtering tools is tdfilter. 
It allows filtering of single dumps out of a whole load of dumps.
You may specify one or multiple dumps you want to be included in the resulting output by specifying the \inlineCode{-x} or \inlineCode{-i} parameters.
The \inlineCode{-x} parameter is used to address a thread by its index (starting with 0), while you address threads by their id when using \inlineCode{-i}.
For example you have these dumps:
\begin{lstlisting}
<dump id="2009-03-06 21:40:43">...</dump>
<dump id="2009-03-06 21:40:44">...</dump>
<dump id="2009-03-06 21:40:45">...</dump>
<dump id="2009-03-06 21:40:47">...</dump>
\end{lstlisting}

Now we want to get the second, third and fourth dumps:
\begin{lstlisting}
tdstrip logfile | tdfilter -x 1 -x 2 -x 3
\end{lstlisting}

results in:
\begin{lstlisting}
<dump id="2009-03-06 21:40:44">...</dump>
<dump id="2009-03-06 21:40:45">...</dump>
<dump id="2009-03-06 21:40:47">...</dump>
\end{lstlisting}

We can also get them by using their ids:
\begin{lstlisting}
tdstrip logfile | tdfilter \
-i "2009-03-06 21:40:44"\
-i "2009-03-06 21:40:45"\
-i "2009-03-06 21:40:47"
\end{lstlisting}

You may've noticed that I selected three following dumps.
Such cases can be solved by specifying ranges.
Both \inlineCode{-i} and \inlineCode{-x} support ranges.
For indices the ranges are defined as \inlineCode{-x n-m} while m is
excluded.
ID ranges are defined as \inlineCode{-i n--m}.
I decided for a different delimiter (-- vs -) as ID may contain single minus characters
and it would've been impossible to use ranges then.
The second difference again index ranges is, that ID ranges include there
ending as ID's must be known to be used.
It wouldn't be natural to think 'I want to select these, so I have to select 
the next higher ID because of the exclusion.'.
Here both examples using ranges:
\begin{lstlisting}
tdstrip logfile | tdfilter -x 1-4
tdstrip logfile | tdfilter -i "2009-03-06 21:40:44"--"2009-03-06 21:40:49"
\end{lstlisting}

Did you notice something?
In the latter example I specified a non-existing ID as end.
As the range means 'everything starting from 2009...44 until reaching 2009...49'
we reach the starting ID but never reach the ending ID.
This means we select all dumps from the starting ID until the end.
