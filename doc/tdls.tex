\subsection{tdls}

To get a list of all threads of a tdstrip output the tool tdls helps you just like ls does.
For example you have a couple of dumps with plenty of threads.
Using tdls it will look like this:
\begin{lstlisting}
[dump_id]
Total threads: n
    [native_pid] [state] [thread_name]
    ...
    
[dump_id]
Total threads: n
    [native_pid] [state] [thread_name]
    ...    
\end{lstlisting}

If you need the java PID, you have to specify the \inlineCode{-j} parameter.
Process ID's are displayed in decimal per default.
If you need their original hexadecimal value, specify the \inlineCode{-H} parameter.

The state is just a one-letter presentation of the state name, which maps as followed:

\begin{tabular}{r | l}
R & Running \\ \hline
O & Object.wait() \\ \hline
C & Waiting on condition \\ \hline
M & Waiting for monitor \\ \hline
S & Sleeping \\ \hline
U & Unknown
\end{tabular}

There are also some parameters to specify the ordering:

\begin{description}
\item[-j] Sort by java PID.
\item[-p] Sort by native PID.
\item[-n] Sort by name (\textbf{default}).
\end{description}

Just like ls you may pass \inlineCode{-r} to display all threads in reversed order.
The dumps are still in their natural order.

As you see tdls enables you to get a brief overview of what threads are inside the dumps and in which states they are.
