\subsection{tdgrep}

The second filtering tool is \inlineCode{tdgrep}.
It works very similar to \inlineCode{grep}.
Filtering works using an expression, a filtering method and a filtering field.
To be more clearly: ``tdgrep returns all threads where the [filtering\_method] successfully applied to [filtering\_field] using [expression]''.

The filtering field is a part of a thread.
This can be one of:
\begin{description}
\item[-n] The thread name.
\item[-t] A stacktrace element (\textbf{default}).
\item[-p] The native thread PID.
\item[-j] The java thread PID.
\end{description}

The filtering method defines how the expression is being applied to the field.
Currently we have these methods:
\begin{description}
\item[-s] The field has to start with the expression..
\item[-c] The field has to contain the expression (\textbf{default}).
\item[-e] The field has to be exactly the expression.
\item[-r] The expression is compiled as regular expression.
\end{description}

To find long running threads you may specify the \inlineCode{-m} parameter.
A long running thread is a thread which occurs in more than one thread dump.
The parameter requires the absolute number of minimum dumps in which the thread has to occur.

For example: If you specify \inlineCode{-m 5} all threads which match the selected
filtering options have to occur in at least five dumps.
