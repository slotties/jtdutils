\subsection{tdlocks}

With tdls you can get an overview of all threads, but what now?
Another tool is tdlocks.
It displays all locks and a couple of information you may need.
The output has this format:
\begin{lstlisting}
Lock: [lock_obj_addr] ([lock_obj_class])
Owners: n
        [thread_name]
        ...
Waiting threads: n
        [thread_name]
        ...
\end{lstlisting}

Each lock information block contains the object address of the locking object and its class.
These information allow you to see what kind of object is used to lock and, if you have a heap dump at the very same time the thread dump was taken, you may check the object by using its address.
Additionaly each block has one section for the lock owner and one for those threads that wait for the lock.
Sometimes no owner is displayed.
In this case no \inlineCode{synchronized}-block is used, but \inlineCode{Object.wait()}.
When no object is currently running and all are waiting for a thread to call \inlineCode{Object.notify()} you won't see any owner.

In a dump can be very many locks.
To handle this you can specify the \inlineCode{-w} parameter so tdlocks just displays locks having at least n waiting threads.
You can also use \inlineCode{-c} to specify the lock object class, if you search for specific locks.
