\section{Build}

\subsection{Preparation}
To build the jtdutils you need to check it out:
\begin{lstlisting}
svn co https://jtdutils.svn.sourceforge.net/svnroot/jtdutils/trunk
\end{lstlisting}

Currently the structure is as followed:
\begin{description}
\item[doc] contains documentation, like this here.
\item[test] contains unit tests.
\item[tools] contains extra stuff, like ZSH completion functions.
\end{description}

Each component or tool of the jtdutils suite has its own directory.

\subsection{Tools}

To build the tools you may switch to the top level directory (\fileName{jtdutils/trunk})
and invoke
\begin{lstlisting}
make
\end{lstlisting}

done! 
There are a couple of more targets:
\begin{description}
\item[man] gzip's and installs man-pages.
\item[libtd] builds the libtd lib.
\item[unit\_tests] invokes the unit tests.
\item[zshcomp] installs the ZSH components.
\item[dist-src] creates the source distribution GZIP.
\item[dist-bin] creates the binary distribution GZIP.
\item[arch] creates an ARCH Linux package.
\end{description}

Each tool has its own target, so it can be build independent.

\subsection{Documentation}

The documentation has its own makefile currently. 
Just cd into \fileName{doc/} and invoke make. 
Currently the only other target, which is automatically called by the all-target, is pdf.
The documentation build requires \fileName{pdflatex} to build.
